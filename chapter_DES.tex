\chapterimage{head2.png} % Chapter heading image
\chapter{The Dark Energy Survey}
\label{ch:DES}

The Dark Energy Survey (DES) \cite{2005astro.ph.10346T} is a $grizY$ photometric galaxy survey that has as main scientific goal to shed light on the nature of the Dark Energy. DES uses four probes to unravel the nature of Dark Energy: the number of clusters as a function of redshift, the measurement of the scale of the baryon acoustic oscillations (BAO) peak, the weak gravitational lensing of galaxies and the measurement of the Hubble diagram with type Ia Supernovae (SNIa). By the end of five years of observations, DES will cover 5000 deg$^2$ of the Southern Hemisphere up to magnitude $i<24.0$ at the $10\sigma$ detection level. Taking this into account, this survey is expected to measure 10000 clusters up to redshift 1.0, 200 million galaxy-shapes for weak-lensing (with $z<1$), 300 million galaxies for BAO ($z<1.4$) and 3000 SNIa up to redshift 1.0. The power of DES resides on the combination of all the probes breaking degeneracies on the cosmological parameter phase-space leading to a precision better than the 5\% on the parameters $w_0$ and $\Delta w_a<0.2$ on the equation of state of the Dark Energy.
\newline

DES is an international collaboration formed by about 500 scientists from more than 20 institutions from: USA, Spain, UK, Brazil, Germany, Australia, Chile and Switzerland. The Collaboration has built a very sensitive camera, DECam (see \autoref{fig:decam} and \autoref{fig:cerrotololo}) \cite{DIEHL20121332}, that has been mounted at the 4-m Victor M. Blanco Telescope\footnote{This is the same telescope where Schmidt and Perlmutter performed some of the observations leading to the Nobel Prize in 2010 for the discovery of Dark Energy.} at the Cerro Tololo Inter-American Observatory (CTIO), located near La Serena (Chile).

\section{The DECam}
The DECam (Dark Energy Camera), is the main instrument of the experiment. It is composed mainly by:
\begin{itemize}
	\item The 570 megapixel focal plane, formed by 70 CCDs.
    \item Low-noise readout electronics.
    \item Wide-field optical corrector, producing 3 deg$^2$ field of view.
    \item Filter and shutter system.
    \item Hexapod for stability.   
\end{itemize}
Since DES is going to observe very highly redshifted galaxies, the camera's CCDs have been specifically designed at Lawrence Berkeley National Laboratory to detect red light. In order to do so, the silicon layer is ten times thicker --250 $\mu m$-- than conventional ones\footnote{Sensitivity to long wavelengths is increased when passing trough more silicon.}. This results in a quantum efficiency $>$80\% on the 600-950 nm range, $>$60\% on the 400-600 nm and $>$50\% on the 900-1000 nm. The DECam focal plane consist of the following types of CCDs:
\begin{itemize}
\item Science array: formed by 62 CCDs with $2048\times 4096$ pixels. Each pixel is $15\mu m$ of side that, at the prime focus fot the  Blanco Telescope, results on a pixel scale of 0.27 arc-seconds on the sky.
\item Four $2048\times2048$ guider CCDs.
\item Eight $2048\times 2048$ focus and alignment CCDs.
\end{itemize}
To minimize the noise and dark currents due to the electronics system, DECam operates on an environment cooled by liquid nitrogen at 180 K and a vacuum of $\sim 10^{-9}$ atm. The whole readout process takes 17 seconds (about the same as the slewing-time of the telescope). Readout and clocking electronic boards were produced and designed in Spain at CIEMAT and IFAE.
\begin{figure}
\begin{center}
\includegraphics[width=0.9\textwidth]{./Pictures/telescope_DES_mine.jpg}
\caption{DECam mounted at the focus of the Victor Blanco Telescope. Image credit: M. Garcia-Fernandez}
\label{fig:decam}
\end{center}
\end{figure}

\section{Survey strategy}
The total amount of time awarded to DES at CTIO for observing the total area to the nominal depth on the five photometric bands is of 525 nights over a 5-year period. The rest of the nights, DECam is available to the scientific community. The tank-shaped footprint, that can be seen at \autoref{fig:des_footprint} is not casual but is optimized for the several probes.
\begin{itemize}
	\item The {\it cannon} located at the equator, is known as stripe-82 and overlaps with several spectroscopic surveys such as SDSS, to calibrate the photometric redshifts (photo-z hereafter).
    \item The rounded shape --{\it the body}-- is intended to have the largest possible scales for BAO measurement.
    \item The lower part --{\it the wheels}-- is designed to overlap with the South Pole Telescope (SPT) to measure the Sunyaev-Zel'dovich effect correlations with CMB.
\end{itemize}

\begin{figure}
\begin{center}
\includegraphics[width=\textwidth]{./Pictures/cerrotololo_mine.jpg}
\caption{Location of the 4-m Victor Blanco Telescope at Cerro Tololo. Chilean Andes. Image credit: M. Garcia-Fernandez}
\label{fig:cerrotololo}
\vspace{2cm}
\includegraphics[width=\textwidth]{des_footprint.png}
\caption{DES footprint on equatorial coordinates. Purple area is the total area that DES will cover at the end of the five years (Y5). Red areas -that overlap with the purple- are the Science Verification observations. Shaded areas are the first year campaign of observations (Y1). Dark blue regions are the SNIa fields. Dotted line represent the galactic plane. Image credit: The DES Collaboration.}
\label{fig:des_footprint}
\end{center}
\end{figure}

The DES observations can be split in two: the transient survey and the wide-field survey.

\subsection*{The transient survey}
The transient survey is designed to measure SNIa. Selected small portions of the sky --known as the supernovae fields-- are surveyed periodically to time to look for supernovae explosions and measure its luminosity curve as a function of time. Although it is designed to SNIa astronomy, some ancillary Solar-System astronomical results have been reported, such as Jupiter-trojans and trans-Neptunian detection and searches for the recently proposed planet-9 \cite{2016AJ....151...22B}. These ancillary physics also use the wide-field to increase the area.

\subsection*{The wide-field survey}
The wide field survey is intended for the rest of the dark energy probes. The observed area is visited 10 times on each band along the 5-year-period to reach the full depth and the maximum level of homogeneity.
\subsection*{Nightly operations at CTIO}
A typical night of observations, if the sky is not overcast and no earthquake threatens the life of the observers, starts in the afternoon taking calibration data on CCDs. Then, after the evening twilight, three standard stars are imaged to calibrate the photometry. These, are well known stars with very well defined and measured photometric properties. After that, the wide-field survey starts. When the supernovae fields are visible --and time requirements are fulfilled--, they are surveyed, returning to the wide-field survey when they are done. Some time before the morning twilight, another three standard stars are imaged, finishing the night. All nightly operations follows the same pattern except if some transient alarm us received. In this case,  DES points to the place where the transient has been produced to look for an optical counterpart.

\section{The data reduction pipeline}
The data reduction that goes from images to science-ready catalogs of galaxies is carried out at the NCSA\footnote{National Center for Supercomputing Applications. Illinois (USA).}. The first step is to calibrate the data. Then, the different exposures of the same region of the sky for a given band --single-epoch images-- are combined into a single image on a procedure called co-addition --multi-epoch image--. This procedure allows the increase of the observed depth respect to each individual single-epoch image (\autoref{fig:coadd}). Nevertheless, to reach the DES nominal depth, images are detected on the $r+i+z$ multi-epoch images. This multi-epoch images will constitute the measurement images for each band. Co-addition of the objects is made with the software {\scshape Swarp} \cite{2002ASPC..281..228B} and the detection and photometric measurements is made with {\scshape SExtractor} \cite{1996A&AS..117..393B} in dual mode. {\scshape IM3SHAPE} \cite{2013MNRAS.434.1604Z} and {\scshape NGMIX} \cite{2015ascl.soft08008S} packages are used for specific needs like shapes for shear and precise photometry.
\begin{figure}
\begin{center}
\begin{overpic}[width=0.4\textwidth,trim=0 2cm 0 0,clip]{./Pictures/destile_g.png}\put(5,5){\colorbox{white}{\Large\it g}}
\end{overpic}\hspace*{0.1cm}
\begin{overpic}[width=0.4\textwidth,trim=0 2cm 0 0,clip]{./Pictures/destile_r.png}\put(5,5){\colorbox{white}{\Large\it r}}
\end{overpic}\\
\vspace*{0.2cm}
\begin{overpic}[width=0.4\textwidth,trim=0 2cm 0 0,clip]{./Pictures/destile_i.png}\put(5,5){\colorbox{white}{\Large\it i}}
\end{overpic}\hspace*{0.1cm}
\begin{overpic}[width=0.4\textwidth,trim=0 2cm 0 0,clip]{./Pictures/destile_z.png}\put(5,5){\colorbox{white}{\Large\it z}}
\end{overpic}\\
\vspace*{0.2cm}
\begin{overpic}[width=0.4\textwidth,trim=0 2cm 0 0,clip]{./Pictures/destile_det.png}\put(5,5){\colorbox{white}{\Large coadd}}
\end{overpic}
\caption{Comparison of the multi-epoch image for the {\it griz} bands with the detection coadd. Images are taken from DES-database for a region of the tile DES0419-4914 after the Y1 epoch. Image credit: M. Garcia-Fernandez \& The DES Collaboration.}
\label{fig:coadd}
\end{center}
\end{figure}

\section{Current status and latest results}
The Dark Energy Survey began its journey in 2005 with the construction of DECam, starting the data acquisition on 2012 with the Science Verification period. By the end of February 2017, the Year 4 observation campaign has ended (\autoref{fig:des_coverage}). The Year 3 reduction pipeline from images to galaxy-catalogs has just finished and is still under inspection, so the most recent data-set that is being used for Cosmology analysis, is the Year 1 release (\autoref{fig:des_y1_coverage} and \autoref{fig:des_y1_mag_auto_i}).
\newline

Currently, no precise constrains on dark energy have been made yet, since they require an extensive and demanding control of systematic errors that is still ongoing. Nevertheless several other works on Cosmology have been provided, such as strong-lensing \cite{2015MNRAS.454.1260A,2016ApJ...827...51N,2017ApJ...838L..15L}, Sunyaev-Zel'dovich \cite{2016arXiv160508770S}, voids and troughs \cite{2016MNRAS.455.3367G,2017MNRAS.465..746S}, tests of log-normality \cite{2017MNRAS.466.1444C}, clusters \cite{2017MNRAS.467.4015H}, weak-lensing \cite{2015PhRvD..92b2006V,2016PhRvD..94b2002B,2016MNRAS.459...21K,2016MNRAS.461.3172S,2016MNRAS.461.4099B,2016arXiv160908167P,2017MNRAS.465.4204C} and large-scale-structure correlations with CMB \cite{2016MNRAS.456.3213G,2016MNRAS.459...21K,2017MNRAS.465.4166K}.
\newline

Constrains on the cosmological parameter space  provided by DES are based on the Science Verification shear analysis \cite{2016PhRvD..94b2001A,2016MNRAS.463.3653K}. Nevertheless, the most powerful measurement is produced by the combination of clustering with gg-lensing \cite{2017MNRAS.464.4045K} on the $\sigma_8-\Omega_M$ plane. Although results provided are not yet competitive, it is a remarkable milestone for DES to provide such results with just the 3\% of the planed total area (\autoref{fig:des_lcdm}).
\begin{figure}
\begin{center}
\includegraphics[width=\textwidth]{./Pictures/des_tiles.png}
\caption{DES coverage at the end of Year 4 observations campaign for the {\it grizY} photometric bands. All the area has at least 6 tiles out of 10. It can be also seen that there is plenty of area with 7 tiles. Image credit: The DES Collaboration.}
\label{fig:des_coverage}
\end{center}
\end{figure}
\begin{figure}
\begin{center}
\includegraphics[width=0.9\textwidth]{./Pictures/des_y1_coverage.png}
\caption{DES Year 1 spatial distribution of objects on equatorial coordinates. Image credit: The DES Collaboration.}
\label{fig:des_y1_coverage}

\includegraphics[width=0.9\textwidth]{./Pictures/des_y1_mag_auto_i.png}
\caption{DES Year 1 magnitude distribution of objects on the $i$-band (arbitrary normalization). The average depth reached is $i\sim 22.8$. Image credit: The DES Collaboration.}
\label{fig:des_y1_mag_auto_i}
\end{center}
\end{figure}
\newline

But not everything is about dark energy at DES. Several other results has been provided \cite{2016MNRAS.460.1270D}: discovery of several trans-neptunian objects (TNOs), Jupiter-trojans and main belt asteroids, characterization of variable stars, detection and characterization of Milky-Way satellite galaxies --and its use to put constrains on dark matter-- and gravitational-wave follow-up.

\begin{sidewaysfigure}
\includegraphics[width=\textwidth]{./Pictures/des_lcdm.png}
\caption{DES-SV constrains on the $\Omega_M-\sigma_8$ plane combining clustering and gg-lensing \cite{2017MNRAS.464.4045K}. Two scenarios are considered: $\Lambda$CDM and the presence of non-evolving dark energy ($w$CDM).}
\label{fig:des_lcdm}
\end{sidewaysfigure}