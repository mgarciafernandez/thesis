\chapter*{Summary}
\addcontentsline{toc}{chapter}{\color{ocre}Summary}
Cosmological measurements show that the expansion of the Universe is accelerating. Generically, the entity that causes this acceleration is called dark energy. Nevertheless, the nature of the dark energy constitutes one of the biggest puzzles in Physics. Shedding light on dark energy requires the construction of experiments able to survey large volumes of the Universe. One of those experiments is the Dark Energy Survey (DES) in which this thesis has been developed.
\newline

One of the observational probes that may unravel the nature of dark energy is the weak gravitational lensing. Weak-lensing is produced by the bending of the trajectory of photons by gravitational fields leading to the deflection of the light rays. Thus, the light emitted by distant galaxies is deflected by the matter located between them and the observer. For extended sources, in addition to the change in position, this leads to two observational effects: an isotropic size enlargement (magnification) and an elongation/shrink along one axis (shear). Since the surface brightness is preserved, the isotropic size enlargement due to magnification produces an increase on the observed flux of the background galaxies. This fact allows to see galaxies that would be beyond the detection threshold if gravitational lensing were not present. Thus, close to the lenses, the observed density of sources is increased. This effect is known as number-count magnification and allows to probe the convergence profile of the selected lens sample, that is a proxy for the matter profile.
\newline

This Thesis is devoted to the analysis of weak-lensing magnification on the Dark Energy Survey. Two analysis with different goals each are made on different data-sets: the Science Verification (DES-SV) and the Year 1 (DES-Y1). The DES-SV analysis aims the development of techniques to detect the weak-lensing number count magnification signal and the mitigation of systematic errors. The DES-Y1 analysis employs the methods used at the DES-SV data to measure the convergence profile of the emptiest regions of the Universe --voids and troughs-- to use them as a new cosmological probe.