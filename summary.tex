\chapter*{Summary}
\addcontentsline{toc}{chapter}{\color{ocre}Summary}
Latest measurements from different experiments lead to two observational effects: the Universe is flat, and its expansion is accelerated. Previos results are not allowed at the same time with the current gravitational theory --General Relativity-- on its original formulation. The alternatives to General Relativity to take into account the accelerated expansion on flat universes are: the cosmological constant, the presence of exotic quantum fields and modified gravity theories. Whatever it's the answer it is called dark energy.
\newline

The current experiment specifically designed to unravel the nature of dark energy is the Dark Energy Survey (DES), that will use four probes to discriminate which theory is right: Type Ia Supernovae (SNIa), cluster counts, barion acoustic oscillation (BAO) and weak-lensing.
\newline

Weak-lensing is produced by the gravitational bending of the trajectory of photons by gravitational fields leading to the deflection of the light rays. Thus, the light emitted by foreground distant galaxies is deflected by the matter located between them and the observe. For extended sources, in addition to the change in position, for extended sources, this leads to two observational effects: an isotropic size enlargement (magnification) and an elongation/shrink along one axis (shear).
\newline

Since the surface brightness is preserved, the isotropic size enlargement due to magnification produces an increase on the observed flux of the background galaxies. This allows to see galaxies that would be beyond the detection threshold if gravitational lensing was not present. Since this flux augmentation is dependent on the distance to the objects that produce lensing, nearby the lenses the observed density of sources is increased. This effect is known as number-count magnification and allows to probe the convergence profile of the lens sample selected, that is a proxy for the matter profile.
\newline

On this Thesis, a methodology to study number-count magnification is developed and applied to the Dark Energy Survey Science Verification data. This new methodology employs source galaxies from the general population selected purely by photometric redshift. This sample is much more numerous, allowing to use a sample with much lower number of lenses. In addition a new technique to estimate systematic errors using simulations has been used, allowing to unbiased an reliable measurements. In additions, on the DES Year 1 data, the convergence profile of voids an troughs is determined using this new methodology.
\newline

The measurement of voids and troughs convergence profile is an excellent way to test the nature of dark energy, since they are large under-dense regions of the Universe and they are environments whose evolution is dominated by dark energy. Nevertheless, theoretical work on the convergence profile of voids is still not available. Thus a new window to test dark energy has been opened with this study that still needs to keep development.