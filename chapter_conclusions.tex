
\chapterimage{head2.png} % Chapter heading image
\chapter{Conclusions}
General Relativity has been the gravitational theory since Einstein conceived it a century ago. Since then, it passed successfully the most stringent tests. Nevertheless, the discovery of the accelerated expansion of the Universe --dark energy-- along with the latest LHC results on High Energy Physics suggest that there must be something beyond General Relativity or the Standard Model of Particle Physics.
\newline

Tests of gravity on cosmological scales can provide an insight on the nature of dark energy. One of those scenarios, are the empties regions of the Universe: voids and troughs. Since they are mostly empty of matter, their evolution and structure is dominated by dark energy. Thus, they constitute a promising environment to test the nature of dark energy.
\newline

Measurements of voids and troughs properties can be made with weak gravitational lensing, namely: magnification and gg-lensing. The advantage of using these two methods is that they are complementary effects of the same physical phenomena but are sensitive to different systematic effects. Thus, the combination of these two weak-lensing methods to measure voids and trough profiles provide an accurate and reliable probe for the nature of dark energy.
\newline

Although wide-field surveys has provided the last years numerous weak-lensing results, magnification has been little studied due to its low signal-to-noise ratio compared with gg-lensing and to its sensitivity to systematic effects.
\newline

On this work, a technique to measure magnification with the number-count technique has been presented. In addition, a through and new way to take into account systematic errors has been presented, providing un-biased and reliable measurement.
\newline

Nevertheless, number-count magnification is not the measurement {\it per se}, but a proxy to the convergence profile of the lenses, the final physical observable. This implies that other types of magnification measurements can be made as a proxy to the convergence: the magnitude and size shift.