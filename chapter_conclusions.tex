\chapterimage{head2.png} % Chapter heading image
\chapter{Conclusions}
\label{ch:conclusions}
In this Thesis, the weak-lensing magnification has been measured using two data-sets for the Dark Energy Survey: the Science Verification (DES-SV) and the Year 1 (DES-Y1), with different goals each. The DES-SV analysis developed a methodology to measure and mitigate the impact of systematic errors on number count magnification with wide-field photometric surveys. On the other hand, the DES-Y1 analysis used the techniques employed at DES-SV to measure the convergence profile of voids and trough to provide constrains on cosmology. This Thesis is a result of the active participation on The DES Collaboration and produced, up to date, two publications: a paper \cite{2016arXiv161110326G} and a conference proceeding \cite{2017hsa9.conf..163G}.
\newline

The nature of dark energy --the responsible of the accelerated expansion of the Universe--, constitutes one of the biggest puzzles of Modern Cosmology. The emptiest regions of the Universe --voids and troughs-- are dominated by dark energy. Thus, its structure and evolution constitutes a powerful probe to shed light on dark energy. One of the physical observables from voids and troughs that is sensitive to dark energy is their convergence profile, that is a direct proxy for the matter profile. Due to the presence of dark matter, the matter profile is only directly accessible with gravitational lensing, since the other probes require the parametrization on how galaxies assembly within dark matter.
\newline

Weak-lensing magnification provides a direct measurement of the convergence profile. There are three observables that allow the determination of the convergence profile: number count magnification, size magnification and flux/magnitude magnification. These three methodologies can be combined to provide more accurate measurements and mitigate systematic errors. In addition, magnification can be combined with gg-lensing measurements to improve the systematic error correction since the gg-lensing also probes the convergence profile, but the sources of systematic errors are different.
\newline

In this work, a methodology to correct the effect of systematic errors using two kinds of simulations (N-body and image simulations) has been developed. The most important systematic errors to be taken into account are the observing conditions of the survey and the photo-z overlap of the lenses and sources. To overcome the survey observing conditions, the {\scshape Balrog} image-simulation has been used for the first time. The use of {\scshape Balrog} has demonstrated to provide reliable and unbiased corrections for these kind of systematic errors. In addition, this simulation maps depth inhomogeneities, allowing to use the full depth of the survey, increasing the area and number of available galaxies. On the other hand, the impact photo-z overlap was estimated with the help of the MICE N-body simulation and then hard photo-z cuts where made requiring a negligible impact on the magnification signal.
\newline

Finally, on this work the convergence profile of voids an troughs has been measured for the first time using number count magnification. This measurement opens the window to a new class of cosmological probes, since weak-lensing magnification by voids and troughs constitutes a new and independent probe for dark energy that on the immediate future may provide competitive and reliable cosmological constrains.

