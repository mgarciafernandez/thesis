\chapterimage{head2.png} % Chapter heading image
\chapter{Conclusions}
\label{ch:conclusions}
In this Thesis, the weak-lensing magnification has been measured using two data-sets for the Dark Energy Survey: the Science Verification (DES-SV) and the Year 1 (DES-Y1) with different goals each. The DES-SV analysis developed a methodology to measure and mitigate the impact of systematic errors on number count magnification with wide-field photometric surveys. On the other hand, the DES-Y1 analysis used the techniques employed at DES-SV to measure the convergence profile of voids and trough. This Thesis is a result of the active participation on The DES Collaboration and produced two publications: a paper \cite{2016arXiv161110326G} and a conference proceeding \cite{2017hsa9.conf..163G}.
\newline

The nature of dark energy constitutes one of the biggest puzzles of Modern Cosmology and it is responsible of the accelerated expansion of the Universe. The emptiest regions of the Universe --voids and troughs-- are dominated by dark energy. Thus, its structure and evolution constitutes a powerful probe to shed light on dark energy. The physical observable