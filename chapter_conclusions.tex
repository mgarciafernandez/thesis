
\chapterimage{head2.png} % Chapter heading image
\chapter{Conclusions}
\label{ch:conclusions}
General Relativity has been the gravitational theory since Einstein conceived it a century ago. Since then, it passed successfully the most stringent tests. Nevertheless, the discovery of the accelerated expansion of the Universe --dark energy-- along with the latest LHC results on High Energy Physics suggest that there must be something beyond General Relativity or the Standard Model of Particle Physics.
\newline

Tests of gravity on cosmological scales can provide an insight on the nature of dark energy. One of those scenarios, are the empties regions of the Universe: voids and troughs. Since they are mostly empty of matter, their evolution and structure is dominated by dark energy. Thus, they constitute a promising environment to test the nature of dark energy.
\newline

Measurements of voids and troughs properties can be made with weak gravitational lensing, namely: magnification and gg-lensing. The advantage of using these two methods is that they are complementary effects of the same physical phenomena but are sensitive to different systematic effects. Thus, the combination of these two weak-lensing methods to measure voids and trough profiles provide an accurate and reliable probe for the nature of dark energy.
\newline

Although wide-field surveys has provided the last years numerous weak-lensing results, magnification has been little studied due to its low signal-to-noise ratio compared with gg-lensing and to its sensitivity to systematic effects.
\newline

On this work, a technique to measure magnification with the number-count technique has been presented. In addition, a through and new way to take into account systematic errors has been presented, providing un-biased and reliable measurement.
\newline

Nevertheless, number-count magnification is not the measurement {\it per se}, but a proxy to the convergence profile of the lenses, the final physical observable. This implies that other types of magnification measurements --giving robustenss to the measurement-- can be made as a proxy to the convergence: the magnitude and size shift.
\newline

Weak-lensing measurements are conceived as one of the four key probes for dark energy to be combined within the Dark Energy Survey. Nevertheless, magnification has shown to be very sensitive to systematic effects that complicate severely the combination of this measurement on a multi-probe fit. This does not imply that magnification can not be used as a competitive probe for Cosmology. The use of weak-lensing magnification as a stand-alone or in combination with gg-lensing on low-scale studies or extreme environments where dark energy dominates, constitute independent and alternative probes for gravity.
\newline

Unfortunately as of the day this Thesis started to be written --March 2017-- no theoretical expression of the convergence profile of voids with non-cosmological-constant dark energy models is available. Nevertheless, it is known how to proceed: currently available General Relativity LTB void profiles can be assumed as {\it a posteriori} solution on LTB $f(R)$ modified gravity models. This leads to a subset of $f(R)$ models that can be constrained. Then, with this models, angular diameter distances can be computed and from them the convergence profile. This leads to a physical observable that allow to discriminate between General Relativity and an specific subset of modified gravity models, task that will be addressed on the near future.
\newline

Although, the determination of void profiles with magnification constitutes a promising tool for the gravitational theory, other questions on Cosmology can be answered with weak-lensing magnification, such as the large-scale-structure of the Universe. Matter profile of dark matter halos can be measured on both galaxies and clusters, allowing to answer questions such as the nature of dark matter or the halo-bias connection.
\newline

The new way to take into account systematic effects developed on this thesis, stablish weak-lensing magnification as a robust, reliable, unbiased and competitive cosmological probe, that opens a new way to explore the Cosmos giving light on the dark Universe: dark matter and  dark energy.