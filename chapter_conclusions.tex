\chapterimage{head2.png} % Chapter heading image
\chapter{Conclusions}
\label{ch:conclusions}
In this Thesis, the weak-lensing magnification has been measured using two data-sets for the Dark Energy Survey: the Science Verification (DES-SV) and the Year 1 (DES-Y1) with different goals each. The DES-SV analysis developed a methodology to measure and mitigate the impact of systematic errors on number count magnification with wide-field photometric surveys. On the other hand, the DES-Y1 analysis used the techniques employed at DES-SV to measure the convergence profile of voids and trough. This Thesis is a result of the active participation on The DES Collaboration and produced two publications: a paper \cite{2016arXiv161110326G} and a conference proceeding \cite{2017hsa9.conf..163G}.
\newline

The accelerated expansion of the Universe constitutes one of the biggest puzzles of Modern Cosmology. Unravel the nature of dark energy requires the combination of different probes to break degeneracies on the cosmological parameters. One of those probes is the weak gravitational lensing.
\newline

The wea