\chapter*{Resumen y Conclusiones}
\addcontentsline{toc}{chapter}{\color{ocre}Resumen y Conclusiones}
\section*{Resumen}
\addcontentsline{toc}{section}{Resumen}
Las medidas cosmol\'ogicas muestran que la expansi\'on del universo se est\'a acelerando. Gen\'ericamente, el agente causante de esta aceleraci\'on se llama energ\'ia oscura. Sin embargo, la naturaleza de la energ\'ia oscura constituye uno de los grandes misterios de la F\'isica. Arrojar luz sobre la naturaleza de la energ\'ia oscura requiere la construcci\'on de experimentos que cartograf\'ien grandes vol\'umenes del universo. Uno de dichos experimentos es el {\it Dark Energy Survey} (DES) en el cual esta tesis ha sido desarrollada.
\newline

Entre las sondas observacionales que pueden desvelar la naturaleza de la energ\'ia oscura se encuentran las lentes gravitacionales d\'ebiles. El efecto lente gravitacional se produce al curvarse la trayectoria de los fotones por los campos gravitatorios, produciendo la deflexi\'on de los rayos de luz. Entonces, la luz emitida por galaxias fuente lejanas es desviada por la materia localizada entre ellas y el observador. En el caso de fuentes extensas, a mayores del cambio en la posici\'on, esto produce dos efectos observacionales: un aumento is\'otropo del tama\~no (magnificaci\'on) y una elongaci\'on/contracci\'on a lo largo de un eje ({\it shear}). Dado que el brillo superficial se conserva, el aumento is\'otropo del tama\~no debido a la magnificaci\'on produce un incremento del flujo observado en las galaxias fuentes. Esto permite ver galaxias que estar\'ian por debajo del umbral de detecci\'on si el efecto de lente gravitacional no existiese. Por tanto, cerca de las lentes la densidad de fuentes observada se incrementa. Este efecto se conoce como {\it number-count magnification} y permite medir el perfil de convergencia de la muestra seleccionada como lente, que es directamente dependiente del perfil de materia.
\newline

Esta tesis est\'a dedicada al an\'alisis de la magnificaci\'on por lentes gravitacionales d\'ebiles en el Dark Energy Survey. Se hacen dos an\'alisis distintos en dos muestras de datos diferentes y con diferentes objetivos: el {Science Verification} (DES-SV) y el {\it Year 1} (DES-Y1). El an\'alisis realizado en DES-SV tiene como objetivo el desarrollo de t\'ecnicas para detectar la se\~nal del efecto {\it number-count magnification} y la correcci\'on de sus errores sistem\'aticos. El an\'alisis de DES-Y1 emplea los m\'etodos desarrollados en la muestra DES-SV para medir el perfil de convergencia de las regiones m\'as vac\'ias del universo --{\it voids} y {\it troughs}-- para usarlas como una nueva sonda cosmol\'ogica.

\section*{Conclusiones}
\addcontentsline{toc}{section}{Conclusiones}
En esta tesis, el efecto de magnificaci\'on por lentes gravitacionales d\'ebiles ha sido medido usando dos muestras de datos del Dark Energy Survey: el {\it Science Verification} (DES-Y1) y el {\it Year 1} (DES-Y1), con diferentes objetivos cada una. En el an\'alisis realizado con el DES-SV se han desarrollado t\'ecnicas para medir el efecto y mitigar el impacto de posibles errores sistem\'aticos en la se\~nal de {\it number-count magnification} con cartografiados fotom\'etricos de gran campo. Por otro lado, el an\'alisis del DES-Y1 ha hecho uso de las t\'ecnicas desarrolladas en el DES-SV para medir el perfil de convergencia de {\it voids} y {\it troughs} para proporcional medidas cosmol\'ogicas.
\newline

La naturaleza de la energ\'ia oscura --la responsable de la expansi\'on acelerada del universo--, constituye uno de los grandes misterios de la cosmolog\'ia moderna. Las regiones m\'as vac\'ias del cosmos --{\it voids} y {\it troughs}-- est\'an dominadas por la energ\'ia oscura, por lo que su estructura y evoluci\'on constituye un m\'etodo muy potente para estudiar su naturaleza. Uno de los observables f\'isicos de {\it voids} y {\it troughs} que es sensible a la energ\'ia oscura es su perfil de convergencia, que es una estimaci\'on directa del perfil de materia. Debido a la presencia de materia oscura, el perfil de materia es s\'olamente accesible a trav\'es de las lentes gravitacionales, dado que los otros m\'etodos requieren de la parametrizaci\'on de c\'omo las galaxias se relacionan con la materia oscura.
\newline

La magnificaci\'on por lentes gravitacionales proporciona una medida directa del perfil de convergencia. Hay tres observables que permiten determinar el perfil de convergencia: {\it number-count magnification}, la magnificaci\'on del tama\~no y la magnificaci\'on del flujo/magnitud. Estos tres m\'etodos pueden ser combinados para proporcionar medidas m\'as precisas y mitigar los efectos de errores sistem\'aticos. Adem\'as, la magnificaci\'on puede ser combinada con medidas de {\it gg-lensing} para mejorar la correci\'on de los errores sistem\'aticos dado que el {\it gg-lensing} es tambi\'en sensible al perfil de convergencia, pero sus errores sistem\'aticos son distintos.
\newline

Entre los efectos sistem\'aticos m\'as importantes a tener en cuenta se encuentran las condiciones de observaci\'on y el solapamiento por la determinaci\'on fotom\'etrica del corrimiento al rojo de las galaxias. Para mitigar los errores inducidos por las condiciones de observaci\'on, se ha usado por primera vez la simulaci\'on de im\'agenes {\scshape Balrog}. El uso de {\scshape Balrog} ha demostrado proporcionar correciones fidedigas y no-sesgadas para este tipo de errores sistem\'aticos. Adem\'as, estas simulaciones permiten trazar las inhomogeneidades en la profundidad, permitiendo alcanzar la profundidad completa del experimento, aumentando as\'i el n\'umero de galaxias disponibles. Por otro lado, el impacto del solapamiento por la determinaci\'on fotom\'etrica del corrimiento al rojo ha sido estimado con ayuda de la simulaci\'on de N-cuerpos MICE, realizando luego cortes estrictos en el corrimiento al rojo, requiriendo un impacto despreciable en la se\~nal de magnificaci\'on.
\newline

Finalmente, se ha medido por primera vez el perfil de convergencia de {\it voids} y {\it troughs} usando la magnificaci\'on. Esta medida abre una nueva ventana a una nueva clase de sondas cosmol\'ogicas, puesto que la magnificaci\'on producida por {\it voids} y {\it troughs} constituye una medida nueva e independiente para la energ\'ia oscura, que en un futuro inmediato puede proporcionar medidas cosmol\'ogicas competitivas y fidedignas.