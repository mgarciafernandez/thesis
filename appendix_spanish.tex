\chapter*{Resumen y Conclusiones}
\addcontentsline{toc}{chapter}{\color{ocre}Resumen y Conclusiones}
\section*{Resumen}
\addcontentsline{toc}{section}{Resumen}
Medidas recientes de diferentes experimentos arrojan dos efectos observacionales: el Universo es plano y se expande de forma acelerada. Estes dos resultados no est\'an permitidos simult\'aneamente en la teoria gravitatoria actual --la Relatividad General-- en su formulaci\'on original. Las alternativas a la Relatividad General que tienen en cuenta la expansi\'on acelerada en universos planos son: la constante cosmol\'ogica, la presencia de campos cu\'anticos ex\'oticos y teor\'ias de gravedad modificada. Sea cual sea la respuesta se le denomina energ\'ia oscura.
\newline

El experimento que se est\'a desarrollando actualmente que est\'a espec\'ificamente dise\~nado para desvelar la naturaleza de la energ\'ia oscura es el Dark Energy Survey (DES), que usar\'a cuatro m\'etodos para discriminar qu\'e teor\'ia es la correcta: Supernovas de tipo 1a (SNIa), el n\'umero de cl\'usteres de galaxias, las oscilaciones ac\'usticas de bariones y las lentes gravitacionales d\'ebiles.
\newline

Las lentes gravitacionales d\'ebiles son producidas por la deflexi\'on las trayectorias de los fotones en la presencia de campos gravitatorios lo que se traduce en un curvado de los rayos de luz. Esto implica que la luz emitida por galaxias lejanas es desviada por la materia localizada entre dichas galaxias y el observador. Para fuentes extensas, esto se traduce en dos efectos observacionales: un aumento is\'otropo del tamaño (magnificaci\'on) y una elongaci\'on/encogimiento a lo largo de uno de los ejes ({\it shear}).
\newline

Dado que el brillo superficial se conserva, el incremento del tama\~no debido a la magnificaci\'on produce un incremento del flujo de las galaxias que se encuentran m\'as alejadas. Esto permite ver galaxias que estar\'ian por debajo del umbral de detecci\'on is el efecto de lente gravitacional no existiese. Este efecto es conocido como {\it number-count magnification} y permite medir el perfil de convergencia de la muestra seleccionada como lente, que est\'a directamente relacionado con el perfil de materia.
\newline

En esta Tesis, se desarrolla una nueva metodolog\'ia para estudiar el efecto {\it number-count magnification} y se aplica al cat\'alogo de datos {\it Science Verification} del experimento {\it Dark Energy Survey}. Esta nueva metodolog\'ia empleada usa galaxias fuente de la poblaci\'on general de galaxias seleccionadas puramente por su {\it redshift} fotom\'etrico. Esta muestra est\'a mucho m\'as poblada, lo que permite usar lentes menos densas. Adem\'as una nueva t\'ecnica para estudiar errores sistem\'aticos con la ayuda de simulaciones se ha usado, lo que permite aportar medidas fiables y no sesgadas. Finalmente en los datos {\it Year 1} de DES se ha medido el perfil de convergencia de {\it voids} y {\it troughs} usando esta nueva metodolog\'ia.
\newline

La determinaci\'on del perfil de convergencia de {\it voids} y {\it troughs} es una forma excelente de revelar la naturaleza de la energ\'ia  oscura, dado que son grandes regiones donde hay una gran infra-densidad de materia, por lo que su evoluci\'on y estructura est\'a dominada por la energ\'ia oscura. Sin embargo, las predicciones te\'oricas del perfil de convergencia de {\it voids} en modelos de gravidad modificada no est\'an toda\'ia disponibles. Dado esto, una nueva ventana para desvelar la naturaleza de la energ\'ia oscura se a abierto  aunque a\'un necesita de desarrollo.
\section*{Conclusiones}
\addcontentsline{toc}{section}{Conclusiones}
La Relatividad General ha sido la teor\'ia gravitatoria desde que Einstein la concibiese hace un siglo. Desde entonces, ha pasado satisfactoriamente las pruebas m\'as exigentes. Sin embargo, el descubrimiento de la expansi\'on acelerada del Universo --energ\'ia oscura-- junto con los \'ultimos resultados del LHC en el campo de la F\'isica de Altas Energ\'ias sugiere que algo debe estar mal o en el campo del Model Est\'andar F\'isica de Part\'iculas o que hay algo m\'as all\'a de la Relatividad General.
\newline

Medidas de la gravedad en escalas cosmol\'ogicas puede arrojar pistas sobre la naturaleza de la energ\'ia oscura. Uno de dichos escenarios, son las regiones m\'as vac\'ias del Universo: {\it voids} y {\it troughs}. Dado que dichas regiones est\'an mayormente vac\'ias de materia, su evoluci\'on y estructura est\'a dominada por la energ\'ia oscura. Esto implica que constituyes un entorno prometedor para sondar la energ\'ia oscura.
\newline

Las medidas de las propiedades de {\it voids} y {\it troughs} se pueden hacer con el efecto de lente gravitacional d\'ebil, es decir: magnificaci\'on y {\it gg-lensing}. La ventaja de usar estes dos m\'etodos es que son efectos complementarios del mismo fen\'omeno pero son sensibles a diferentes errores sistem\'aticos. Esto implica que la combinaci\'on de estes dos m\'etodos para medir {\it voids} y {\it troughs} proporciona una medida precisa y fiable para la naturaleza de la energ\'ia oscura.
\newline

Aunque los grandes cartografiados de galaxias han producido durante los \'ultimos a\~nos numerosos resultados de lentes gravitacionales, 